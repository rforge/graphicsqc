\nonstopmode{}
\documentclass[a4paper,oneside]{book}
\usepackage[ae]{Rd}
\usepackage{makeidx}
\makeindex{}
\begin{document}
\chapter*{}
\begin{center}
{\textbf{\huge The `graphicsQC' Package}}
\par\bigskip{\large \today}
\end{center}
\begin{description}
\raggedright{}
\item[Type] \AsIs{Package }
\item[Title] \AsIs{Quality Control for Graphics in \R{}}
\item[Version] \AsIs{0.9 }
\item[Date] \AsIs{2008-07-24 }
\item[Author] \AsIs{Stephen Gardiner }
\item[Maintainer] \AsIs{Stephen Gardiner <sgar060@aucklanduni.ac.nz> }
\item[Depends] \AsIs{XML }
\item[Suggests] \AsIs{Sxslt }
\item[Description] \AsIs{The package provides functions to generate graphics files, compare them with ``model'' files, and report the results. }
\item[License] \AsIs{GPL-2 }
\item[URL] \AsIs{http://graphicsqc.r-forge.r-project.org }
\end{description}
\Rdcontents{\R{} topics documented:}
\HeaderA{graphicsQC-package}{Quality Control for Graphics}{graphicsQC.Rdash.package}
\aliasA{graphicsQC}{graphicsQC-package}{graphicsQC}
\aliasA{graphicsqc}{graphicsQC-package}{graphicsqc}
\keyword{utilities}{graphicsQC-package}
\begin{Description}\relax
Generates graphics files, compares them with ``model'' files,
and reports the results.
\end{Description}
\newpage % I put this here because otherwise the following table goes over the page numbering
\begin{Details}%\relax
\Tabular{ll}{
Package: & graphicsQC\\
Type: & Package\\
Version: & 0.9\\
Date: & 2008-10-30\\
License: & GPL-2\\
}
To generate files, use \code{\LinkA{plotExpr}{plotExpr}},
\code{\LinkA{plotFile}{plotFile}}, or \code{\LinkA{plotFunction}{plotFunction}}.
To compare sets of these, use \code{\LinkA{compare}{compare}}.
To generate a report based on the comparison, use
\code{\LinkA{writeReport}{writeReport}}.

One possible way of using these functions is to create a set of plots in
a directory in an old version of R (say, the control group) using one
of the plotting functions. Then to load a new version of R and create
the same plots in a different directory (say, the test group). A
comparison can then be done by specifying the control and test
directories. Then a report can be made on the comparison object.

It is highly recommended to use separate directories for the test and
control. If the same directory is used for both, all the prefixes in
the test and all the prefixes in the control must be unique, and
auto-detect will not work if the same directory is given twice.
\end{Details}
\begin{Author}\relax
Stephen Gardiner
\end{Author}
\begin{References}\relax
Free Software Foundation, Inc. 2008 \emph{Diffutils}.
\url{http://www.gnu.org/software/diffutils/diffutils.html}

ImageMagick Studio LLC. 2008 \emph{ImageMagick}.
\url{http://www.imagemagick.org/}

Murrell, P. \& Hornik, K. 2003 \emph{Quality Assurance for Graphics in \R}
\url{http://www.ci.tuwien.ac.at/Conferences/DSC-2003/Proceedings/MurrellHornik.pdf}.
\end{References}
\begin{SeeAlso}\relax
\code{\LinkA{plotExpr}{plotExpr}}, \code{\LinkA{plotFile}{plotFile}},
\code{\LinkA{plotFunction}{plotFunction}}, \code{\LinkA{compare}{compare}},
\code{\LinkA{writeReport}{writeReport}}
\end{SeeAlso}
\begin{Examples}
\begin{ExampleCode}
## Not run: 
  # Create some plots to compare (1st and 3rd plots have differences)
  comp1 <- plotExpr(c("plot(1:10)", "plot(4:40)", "x<-3", "plot(2:23)"),
                    c("pdf", "ps"), "myPrefix", "comp1")
  comp2 <- plotExpr(c("plot(1:11)", "plot(4:40)", "x<-3", "plot(5:15)"),
                    c("pdf", "ps"), "myPrefix", "comp2")

  # Compare them
  compExpr <- compare(comp1, comp2)

  # Write a HTML report
  writeReport(compExpr)
## End(Not run)
\end{ExampleCode}
\end{Examples}

\HeaderA{plotExpr}{Plot arbitrary code}{plotExpr}
\aliasA{plotFile}{plotExpr}{plotFile}
\aliasA{plotFunction}{plotExpr}{plotFunction}
\aliasA{plotPackage}{plotExpr}{plotPackage}
\keyword{utilities}{plotExpr}
\begin{Description}\relax
Produce plots from R expression(s), function(s), or file(s) in
specified file formats. An XML file is also created which contains
information about the plots produced.
\end{Description}
\begin{Usage}
\begin{verbatim}
  plotExpr(expr, filetype = NULL, path = NULL, prefix = NULL,
           clear = FALSE)

  plotFile(filename, filetype = NULL, path = NULL, prefix = basename(filename),
           clear = FALSE)

  plotFunction(fun, filetype = NULL, path = NULL, prefix = fun,
               clear = FALSE)
\end{verbatim}
\end{Usage}
\begin{Arguments}
\begin{ldescription}
\item[\code{expr}] character vector of R expressions which may or may not
produce graphical output.
\item[\code{filename}] the name of the file which the expressions are to be
read from. The path is assumed to be relative to the
current working directory unless an absolute path is given.
\item[\code{fun}] character vector naming the function(s) to plot or just the
(named) function.
\item[\code{filetype}] character vector specifying file formats to produce the
plots in (see details for currently supported formats).
\item[\code{path}] character vector; path to produce output in. If not given, the
current working directory will be used.
\item[\code{prefix}] character vector; prefix for files produced. If multiple
functions or files are given, the resulting plotFile or plotFunction
XML file will use the first prefix.
\item[\code{clear}] logical (not NA); remove files with names we might use
first. If \code{clear} is \code{FALSE} and files exist
with names we might use, an error is given.
\end{ldescription}
\end{Arguments}
\begin{Details}\relax
All functions evaluate the code they are given, capturing and recording any
warnings and errors. The code run for plotFunction is extracted from
any example code (see \code{\LinkA{example}{example}}).

If an error is encountered when running a block of code, that
particular block (say, a file or function) will stop being executed
for that filetype but the error will be recorded. If a warning is
encountered, the code will continue being evaluated.

The name for the log file is based on the first prefix and the type of
function producing the plots. For \code{plotExpr} XML logs, 1 log file
will be produced with a name like \sQuote{prefix-log.xml}.
\code{plotFile} and \code{plotFunction} work by making multiple calls
to \code{plotExpr}, so will produce \code{plotExpr} logs (one for each
file or function respectively), as well as their own log, which will
be named (using the first prefix) like \sQuote{prefix-fileLog.xml} and
\sQuote{prefix-funLog.xml} respectively.

Currently supported file formats are \sQuote{pdf}, \sQuote{png}, and
\sQuote{ps}.
\end{Details}
\begin{Value}
A list of class \code{qcPlotExprResult}, \code{qcPlotFileResult}, or
\code{qcPlotFunctionResult} respectively. The list contains
information about the environment creating the plots (Operating
System, R version, date, call), the names of the plots produced and
any warnings/errors produced.
\end{Value}
\begin{Section}{Warning}
Do not give any code that will open a graphics device (especially if
that device is not closed).
\end{Section}
\begin{SeeAlso}\relax
\code{\LinkA{compare}{compare}}, \code{\LinkA{writeReport}{writeReport}}.
\end{SeeAlso}
\begin{Examples}
\begin{ExampleCode}
## Not run: 
  # plotExpr example:
  example1 <- plotExpr(c("plot(1:10)", "plot(4:40)", "x<-3", "plot(2:23)"),
                       c("pdf", "ps"), "example1", "myPrefix")
  # There should now be a folder "example1" in the current
  # working directory containing pdf and ps files and myPrefix-log.xml, ie
  list.files("example1")

  # plotFunction example:
  example2 <- plotFunction(c("plot", "barplot", "hist"), c("pdf", "ps"),
                           path = "example2")
  list.files("example2")

  # A bigger example:
  # require(grid)
  # gridExample <- plotFunction(ls("package:grid"), c("pdf", "png", "ps"),
  #                             path = "gridExample")
## End(Not run)
\end{ExampleCode}
\end{Examples}

\HeaderA{compare}{Compare graphics output}{compare}
\aliasA{compareExpr}{compare}{compareExpr}
\aliasA{compareFile}{compare}{compareFile}
\aliasA{compareFun}{compare}{compareFun}
\keyword{utilities}{compare}
\begin{Description}\relax
Compares plots/warnings/errors from \code{\LinkA{plotExpr}{plotExpr}},
\code{\LinkA{plotFile}{plotFile}}, or \code{\LinkA{plotFunction}{plotFunction}}.
For the text-based formats (i.e. pdf or ps), a .diff file is created.
If ImageMagick is installed, plots of the differences will also be
produced.
\end{Description}
\begin{Usage}
\begin{verbatim}
compare(test, control, path = test$info$directory, 
        erase = c("none", "identical", "files", "all"))
\end{verbatim}
\end{Usage}
\begin{Arguments}
\begin{ldescription}
\item[\code{test, control}] either:
\Itemize{
\item R objects of class \code{qcPlotExpression},
\code{qcPlotFile}, or \code{qcPlotFunction}.
\item Character vectors of the paths to the respective files,
where relative paths are assumed unless an absolute path is given.
\item Character vectors of the directories which contain the log
files to compare. The highest classed object in the folder will
be chosen for comparison (i.e. if a plotFunction log is in the
directory and many plotExpr logs, all of the plotExpr logs will
be assumed to belong to the plotFunction log).
}
The specification for \code{test} and \code{control} can be mixed and
matched, as long as the resulting objects are of the same class.      

\item[\code{path}] character vector; specifies where all the diff
output (.diff files, plots of differences, and comparison logs)
should be placed.

\item[\code{erase}] character vector; one of \code{"none"},
\code{"identical"}, \code{"files"}, or \code{"all"}.
\describe{
\item[\code{"none"}] do not delete anything.
\item[\code{"identical"}] delete plots in the \code{test}
directory which were identical.
\item[\code{"files"}] delete all plots (and .diff files) in the
\code{test} directory (leaving only the log files).
\item[\code{"all"}] delete all files created in the
\code{test} directory and then the directory if it is empty
}
Currently only \code{"none"} is fully supported.

\end{ldescription}
\end{Arguments}
\begin{Details}\relax
Plots are compared using \acronym{GNU} diff. If a difference is
detected and the current filetype being compared is a text-based
format, a .diff file will be produced. If ImageMagick is installed,
plots of differences will also be created.

It is possible for some plots to appear say, in the test group, but
not in the control group (i.e. the function \code{plot} has an extra
example plot in a new version of R). These such plots are classified
as \sQuote{unpaired}. Unpaired files do not have a corresponding plot
to compare with so are separated into an unpaired section. It is also
possible for entire filetypes to be unpaired. Currently if there is a
completely unpaired function or file when trying to compare, recycling
will be used. This is intended to change in the future.

In many instances, it is also useful to know whether there is any
change in warnings or errors. If any difference is detected in the
warnings/errors for a filetype, all of the warnings or errors
(whichever had the difference detected) for that filetype are
given. It is then up to the user to decide what the difference is
(i.e. whether the ordering has changed or if one group has an extra
warning etc.).

For each set of plot-logs being compared, a comparison log will be
produced. So for each pair of qcPlotExprResult logs being compared, a
comparison log will be produced with a name like
\sQuote{testPrefix+controlPrefix-compareExprLog.xml}. When comparing
qcPlotFileResults or qcPlotFunctionResults there will also be a
compareFileLog or compareFunLog produced which will take a name like
\sQuote{testPrefix+controlPrefix-compareFunLog.xml}, where the
testPrefix and controlPrefix are chosen from the first prefixes in the
set of compareExprLogs being compared (which in turn come from the first
plotExpr logs). These logs are placed in \code{path}.
\end{Details}
\begin{Value}
A list of class \code{qcCompareExprResult}, \code{qcCompareFileResult} or
\code{qcCompareFunResult} containing the results of the comparisons.

\code{qcCompareExprResult} files contain a list of info about the Operating
System, R version, date, call, the info from the test, info from the
control, and then information about the results of the comparisons
(results by filetype giving the result, names of diff files and plots of
differences if produced), including any unpaired plots or filetypes
(with corresponding warnings/errors).

For \code{qcCompareFile} or \code{qcCompareFun} an initial info
section is included, followed by a list containing each individual
\code{qcCompareExprResult}.
\end{Value}
\begin{Note}\relax
\acronym{GNU} diff must be installed on the system. ImageMagick is not
necessary, but greatly extends functionality.
\end{Note}
\begin{SeeAlso}\relax
\code{\LinkA{plotExpr}{plotExpr}}, \code{\LinkA{plotFile}{plotFile}},
\code{\LinkA{plotFunction}{plotFunction}}, \code{\LinkA{writeReport}{writeReport}}
\end{SeeAlso}
\begin{Examples}
\begin{ExampleCode}
## Not run: 
  # Create sets to compare (1st and 3rd are different)
  comp1 <- plotExpr(c("plot(1:10)", "plot(4:40)", "x<-3", "plot(2:23)"),
                    c("pdf", "ps"), "myPrefix", "comp1")
  comp2 <- plotExpr(c("plot(1:11)", "plot(4:40)", "x<-3", "plot(5:15)"),
                    c("pdf", "ps"), "myPrefix", "comp2")
  compExpr <- compare(comp1, comp2)
  # All the diff output has been placed in "comp1" (the test directory)
  compExpr
  # For a better way of viewing this, see ?writeReport
## End(Not run)
\end{ExampleCode}
\end{Examples}

\HeaderA{writeReport}{Generate a HTML report based on plots or comparisons}{writeReport}
\keyword{utilities}{writeReport}
\begin{Description}\relax
Will produce a HTML report of the results from any of the
\code{qcPlot*}, or \code{qcCompare*} results.
\end{Description}
\begin{Usage}
\begin{verbatim}
  writeReport(qcResult, xslStyleSheets = NULL)
\end{verbatim}
\end{Usage}
\begin{Arguments}
\begin{ldescription}
\item[\code{qcResult}] one of:
\Itemize{
\item an R object of class \code{qcPlot*}, or \code{qcCompare*}.
\item the path to the log file to report on
\item a path to the directory, where the highest classed log file
will be auto-detected and then reported on (note that first
comparison logs are searched for, then plot logs).
}

\item[\code{xslStyleSheets}] a named list specifying which XSL style sheets
to override by giving the name of the style sheet to override, and
the location of the xsl file. Can override any of:
\dQuote{plotExprStyleSheet}, \dQuote{plotFunAndFileStyleSheet},
\dQuote{compareExprStyleSheet}, and \dQuote{compareFunAndFileStyleSheet}.
If none are specified, the default (system) ones are used.

\end{ldescription}
\end{Arguments}
\begin{Details}\relax
When reporting on an object, all further qcPlot* or qcCompare* files
which the current object refers to are also reported on. This is so
that full information reports can be given, along with individual
breakdowns. In order for this to happen, all log files that
the object currently being reported on refers to must exist, as well
as any subsequent log files that they refer to.

All reports are placed in the same directory as the XML file they
refer to, with the same name, except with the extension changed from
\sQuote{.xml} to \sQuote{.html}.
\end{Details}
\begin{Value}
A character vector giving the (absolute) path of the highest classed
object reported on. Comparison logs are considered higher classed than
plot logs.
\end{Value}
\begin{SeeAlso}\relax
\code{\LinkA{plotExpr}{plotExpr}}, \code{\LinkA{plotFile}{plotFile}},
\code{\LinkA{plotFunction}{plotFunction}}, \code{\LinkA{compare}{compare}}.
\end{SeeAlso}
\begin{Examples}
\begin{ExampleCode}
## Not run: 
  # After running the `?compare' example
  writeReport(compExpr)

  # Showing how to overwrite stylesheets
  # writeReport(compExpr, list(compareExprStyleSheet="~/myCompareExpr.xsl"))
## End(Not run)
\end{ExampleCode}
\end{Examples}

\printindex{}
\end{document}
